\documentclass{report}
\usepackage{graphicx}
\usepackage{cite}
\usepackage{lipsum}
\title{Title of the Document}
\author{Rajath Kiran A}
\date{June 2025}
\begin{document}
	\maketitle
	\tableofcontents
	
	\listoffigures
	\addcontentsline{toc}{chapter}{List of Figures}
	
	\listoftables
	\addcontentsline{toc}{chapter}{List of Tables}
	
	\chapter{Introduction}
	This is the Introduction part of the document where we give a brief overview of the topic.\cite{berry2001introduction} \\
	It explains the background and the motivation behind choosing this subject.
	
	\chapter{Literature}
	This chapter presents a brief overview of previous works and research done in the same field. \\
	It helps us understand what has already been explored and where the knowledge gaps lie.
	
	\chapter{Methodology}
	This chapter explains the methods and tools used to conduct the study or analysis. \\
	It includes any experimental setup, algorithm, or process followed during implementation.
	\begin{figure}[h]
		\centering\includegraphics[width=0.5\textwidth]{vcet_logo.png}
		\caption{Logo of the College}
		\label{Fig:Image}
	\end{figure}
	
	\chapter{Result}
	This chapter provides the outcome of the applied methodology, supported with tables or graphs. \\
	It gives insights into the effectiveness or accuracy of the approach.
	\begin{table}[h]
		\centering
		\begin{tabular}{|c|c|}
			\hline
			Column 1 & Column 2 \\
			\hline
			Value 1 & Value 2 \\
			\hline
		\end{tabular}
		\caption{Sample Result Table}
	\end{table}
	
	\chapter{Conclusion}
	This is the conclusion part where we summarize the findings and observations of the project.\cite{berry2001introduction} \\
	It also mentions possible future improvements or applications of the study.
	
	\bibliographystyle{plain}
	\bibliography{P9}
	\addcontentsline{toc}{chapter}{Bibliography}
\end{document}
